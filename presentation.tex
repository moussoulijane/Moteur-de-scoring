\documentclass[aspectratio=169,11pt]{beamer}

% =============================================================================
% PACKAGES
% =============================================================================
\usepackage[utf8]{inputenc}
\usepackage[T1]{fontenc}
\usepackage[french]{babel}
\usepackage{graphicx}
\usepackage{booktabs}
\usepackage{colortbl}
\usepackage{xcolor}
\usepackage{tikz}
\usepackage{pgfplots}
\usepackage{fontawesome5}
\usepackage{tcolorbox}
\usepackage{multicol}
\usepackage{array}
\usepackage{adjustbox}

% =============================================================================
% THEME ET COULEURS
% =============================================================================
\usetheme{Madrid}
\usecolortheme{default}

% Couleurs personnalisées
\definecolor{bleubank}{RGB}{31,78,121}
\definecolor{bleuclair}{RGB}{91,155,213}
\definecolor{vertok}{RGB}{112,173,71}
\definecolor{orange}{RGB}{237,125,49}
\definecolor{rouge}{RGB}{192,0,0}
\definecolor{gris}{RGB}{128,128,128}
\definecolor{grisclair}{RGB}{242,242,242}

% Appliquer les couleurs au thème
\setbeamercolor{palette primary}{bg=bleubank,fg=white}
\setbeamercolor{palette secondary}{bg=bleuclair,fg=white}
\setbeamercolor{palette tertiary}{bg=bleubank,fg=white}
\setbeamercolor{structure}{fg=bleubank}
\setbeamercolor{title}{fg=white,bg=bleubank}
\setbeamercolor{frametitle}{fg=white,bg=bleubank}
\setbeamercolor{block title}{bg=bleubank,fg=white}
\setbeamercolor{block body}{bg=grisclair,fg=black}

% Pas de navigation
\setbeamertemplate{navigation symbols}{}

% Numéros de page
\setbeamertemplate{footline}[frame number]

% Style des items
\setbeamertemplate{itemize items}[circle]
\setbeamercolor{itemize item}{fg=bleubank}
\setbeamercolor{itemize subitem}{fg=bleuclair}

% =============================================================================
% COMMANDES PERSONNALISÉES
% =============================================================================
\newtcolorbox{kpibox}[2][]{
    colback=grisclair,
    colframe=bleubank,
    fonttitle=\bfseries,
    title=#2,
    #1
}

\newtcolorbox{successbox}[1][]{
    colback=vertok!10,
    colframe=vertok,
    #1
}

\newtcolorbox{warningbox}[1][]{
    colback=orange!10,
    colframe=orange,
    #1
}

% =============================================================================
% INFORMATIONS DU DOCUMENT
% =============================================================================
\title[Classification Réclamations]{\textbf{Classification Automatique des Réclamations Bancaires}}
\subtitle{Rapport de Performance et Analyse d'Impact}
\author{Direction des Réclamations}
\institute{Projet Machine Learning}
\date{\today}

% =============================================================================
% DÉBUT DU DOCUMENT
% =============================================================================
\begin{document}

% -----------------------------------------------------------------------------
% PAGE DE TITRE
% -----------------------------------------------------------------------------
\begin{frame}
    \titlepage
\end{frame}

% -----------------------------------------------------------------------------
% SOMMAIRE
% -----------------------------------------------------------------------------
\begin{frame}{Sommaire}
    \tableofcontents
\end{frame}

% =============================================================================
% SECTION 1: CONTEXTE ET OBJECTIFS
% =============================================================================
\section{Contexte et Objectifs}

\begin{frame}{Contexte du Projet}
    \begin{columns}[T]
        \begin{column}{0.5\textwidth}
            \textbf{\faExclamationTriangle\ Problématique actuelle}
            \begin{itemize}
                \item Traitement manuel des réclamations
                \item Temps de traitement long
                \item Coût élevé par réclamation
                \item Risque d'incohérence des décisions
            \end{itemize}
        \end{column}
        \begin{column}{0.5\textwidth}
            \textbf{\faLightbulb\ Solution proposée}
            \begin{itemize}
                \item Modèle de Machine Learning
                \item Classification automatique
                \item Système à 3 zones de décision
                \item Gain de 169 MAD par réclamation automatisée
            \end{itemize}
        \end{column}
    \end{columns}
\end{frame}

\begin{frame}{Objectifs de Performance}
    \begin{center}
        \begin{tabular}{lcc}
            \toprule
            \textbf{Métrique} & \textbf{Objectif} & \textbf{Importance} \\
            \midrule
            \rowcolor{grisclair} Précision Rejets & $\geq 97\%$ & Éviter clients lésés \\
            Précision Validations & $\geq 95\%$ & Limiter pertes financières \\
            \rowcolor{grisclair} Taux d'Automatisation & 45-55\% & ROI optimal \\
            Temps d'inférence & $< 100$ms & Production temps réel \\
            \bottomrule
        \end{tabular}
    \end{center}
    
    \vspace{0.5cm}
    
    \begin{successbox}
        \centering
        \textbf{Approche à 3 zones:} Rejet auto ($<30\%$) | Audit humain (30-70\%) | Validation auto ($>70\%$)
    \end{successbox}
\end{frame}

% =============================================================================
% SECTION 2: RÉSULTATS DU MODÈLE
% =============================================================================
\section{Résultats du Modèle}

\begin{frame}{Matrice de Confusion - Ensemble de Test}
    \begin{center}
        % REMPLACER PAR VOTRE IMAGE
        \includegraphics[width=0.85\textwidth]{images/1_matrice_confusion_test.png}
    \end{center}
\end{frame}

\begin{frame}{Performance par Famille Produit}
    \begin{center}
        % REMPLACER PAR VOTRE IMAGE
        \includegraphics[width=0.95\textwidth]{images/2_performance_famille_test.png}
    \end{center}
\end{frame}

% =============================================================================
% SECTION 3: ANALYSE D'IMPACT
% =============================================================================
\section{Analyse d'Impact}

\begin{frame}{Impact Total - Base Complète}
    \begin{center}
        % REMPLACER PAR VOTRE IMAGE
        \includegraphics[width=0.9\textwidth]{images/3_impact_total_base_complete.png}
    \end{center}
\end{frame}

\begin{frame}{Impact - Familles avec Accuracy $\geq 95\%$}
    \begin{center}
        % REMPLACER PAR VOTRE IMAGE
        \includegraphics[width=0.95\textwidth]{images/4_impact_95_base_complete.png}
    \end{center}
    
    \vspace{0.3cm}
    
    \begin{successbox}
        \centering
        \faCheckCircle\ \textbf{Recommandation:} Automatiser en priorité ces familles performantes
    \end{successbox}
\end{frame}

\begin{frame}{Impact - Familles avec Accuracy $\geq 90\%$}
    \begin{center}
        % REMPLACER PAR VOTRE IMAGE
        \includegraphics[width=0.95\textwidth]{images/5_impact_90_base_complete.png}
    \end{center}
\end{frame}

\begin{frame}{Familles à Risque - Analyse des Pertes}
    \begin{center}
        % REMPLACER PAR VOTRE IMAGE
        \includegraphics[width=0.95\textwidth]{images/6_familles_pertes_base_complete.png}
    \end{center}
    
    \vspace{0.3cm}
    
    \begin{warningbox}
        \centering
        \faExclamationTriangle\ \textbf{Attention:} Ces familles nécessitent un traitement manuel ou des améliorations
    \end{warningbox}
\end{frame}

% =============================================================================
% SECTION 4: APPROCHE TECHNIQUE
% =============================================================================
\section{Approche Technique}

\begin{frame}{Pourquoi Machine Learning vs Scoring Simple?}
    \begin{columns}[T]
        \begin{column}{0.48\textwidth}
            \textbf{\faTimesCircle\ Scoring Simple}
            \begin{itemize}
                \item Poids arbitraires
                \item Pas d'interactions
                \item Décision binaire forcée
                \item Seuils fixes et rigides
                \item Recalibrage manuel
            \end{itemize}
            
            \vspace{0.3cm}
            \centering
            \textcolor{rouge}{\textbf{Accuracy $\sim$75-80\%}}
        \end{column}
        
        \begin{column}{0.48\textwidth}
            \textbf{\faCheckCircle\ Notre Approche ML}
            \begin{itemize}
                \item Poids appris sur données
                \item Capture les interactions
                \item Zone d'incertitude $\rightarrow$ humain
                \item Seuils optimisés
                \item Réentraînement automatique
            \end{itemize}
            
            \vspace{0.3cm}
            \centering
            \textcolor{vertok}{\textbf{Accuracy $\sim$94\%}}
        \end{column}
    \end{columns}
\end{frame}

\begin{frame}{Architecture du Système de Décision}
    \begin{center}
        \begin{tikzpicture}[scale=0.9]
            % Axe
            \draw[thick, ->] (0,0) -- (12,0) node[right] {Probabilité};
            \draw[thick] (0,-0.2) -- (0,0.2);
            \draw[thick] (12,-0.2) -- (12,0.2);
            
            % Zones
            \fill[rouge!30] (0,0.5) rectangle (3.6,2);
            \fill[orange!30] (3.6,0.5) rectangle (8.4,2);
            \fill[vertok!30] (8.4,0.5) rectangle (12,2);
            
            % Labels zones
            \node at (1.8,1.25) {\textbf{REJET AUTO}};
            \node at (6,1.25) {\textbf{AUDIT HUMAIN}};
            \node at (10.2,1.25) {\textbf{VALIDATION AUTO}};
            
            % Seuils
            \draw[thick, dashed, rouge] (3.6,-0.5) -- (3.6,2.5);
            \draw[thick, dashed, vertok] (8.4,-0.5) -- (8.4,2.5);
            
            % Labels seuils
            \node at (3.6,-0.8) {\textbf{30\%}};
            \node at (8.4,-0.8) {\textbf{70\%}};
            \node at (0,-0.8) {0\%};
            \node at (12,-0.8) {100\%};
            
            % Annotations
            \node[below, rouge] at (1.8,0.3) {\small Haute confiance};
            \node[below, orange] at (6,0.3) {\small Incertain};
            \node[below, vertok] at (10.2,0.3) {\small Haute confiance};
        \end{tikzpicture}
    \end{center}
    
    \vspace{0.5cm}
    
    \begin{center}
        \textit{"Un bon modèle sait ce qu'il ne sait pas"}
    \end{center}
\end{frame}

\begin{frame}{Hyperparamètres Optimisés (Optuna)}
    \begin{center}
        \begin{tabular}{lc}
            \toprule
            \textbf{Paramètre} & \textbf{Valeur Optimale} \\
            \midrule
            \rowcolor{grisclair} max\_depth & 6 \\
            learning\_rate & 0.05 \\
            \rowcolor{grisclair} n\_estimators & 300 \\
            min\_child\_weight & 5 \\
            \rowcolor{grisclair} subsample & 0.80 \\
            colsample\_bytree & 0.80 \\
            \bottomrule
        \end{tabular}
    \end{center}
    
    \vspace{0.5cm}
    
    \begin{block}{Méthode d'optimisation}
        \begin{itemize}
            \item Algorithme: TPE (Tree-structured Parzen Estimator)
            \item Objectif: Maximiser le F1-Score
            \item Nombre de trials: 50+
        \end{itemize}
    \end{block}
\end{frame}

% =============================================================================
% SECTION 5: RECOMMANDATIONS
% =============================================================================
\section{Recommandations}

\begin{frame}{Plan de Déploiement Recommandé}
    \begin{columns}[T]
        \begin{column}{0.32\textwidth}
            \begin{kpibox}{Phase 1: Pilote}
                \begin{itemize}
                    \item Familles $\geq 95\%$ accuracy
                    \item Double vérification
                    \item Durée: 1-2 mois
                \end{itemize}
            \end{kpibox}
        \end{column}
        
        \begin{column}{0.32\textwidth}
            \begin{kpibox}{Phase 2: Extension}
                \begin{itemize}
                    \item Familles $\geq 90\%$ accuracy
                    \item Monitoring renforcé
                    \item Durée: 2-3 mois
                \end{itemize}
            \end{kpibox}
        \end{column}
        
        \begin{column}{0.32\textwidth}
            \begin{kpibox}{Phase 3: Production}
                \begin{itemize}
                    \item Toutes familles éligibles
                    \item Automatisation complète
                    \item Revue mensuelle
                \end{itemize}
            \end{kpibox}
        \end{column}
    \end{columns}
\end{frame}

\begin{frame}{Familles à Prioriser vs À Exclure}
    \begin{columns}[T]
        \begin{column}{0.48\textwidth}
            \begin{successbox}
                \textbf{\faCheckCircle\ À automatiser en priorité}
                \begin{itemize}
                    \item Accuracy $\geq 95\%$
                    \item Volume significatif
                    \item Faibles pertes montant
                    \item Gain net positif
                \end{itemize}
            \end{successbox}
        \end{column}
        
        \begin{column}{0.48\textwidth}
            \begin{warningbox}
                \textbf{\faExclamationTriangle\ À garder en manuel}
                \begin{itemize}
                    \item Accuracy $< 90\%$
                    \item Pertes montant élevées
                    \item Cas complexes (litiges)
                    \item Familles sensibles
                \end{itemize}
            \end{warningbox}
        \end{column}
    \end{columns}
    
    \vspace{0.5cm}
    
    \begin{block}{Principe clé}
        \centering
        \textbf{Automatiser uniquement quand on est sûr, laisser l'humain décider sinon}
    \end{block}
\end{frame}

\begin{frame}{Métriques de Suivi en Production}
    \begin{center}
        \begin{tabular}{llc}
            \toprule
            \textbf{Métrique} & \textbf{Fréquence} & \textbf{Seuil d'alerte} \\
            \midrule
            \rowcolor{grisclair} Accuracy globale & Quotidienne & $< 90\%$ \\
            Taux d'automatisation & Quotidienne & $< 40\%$ ou $> 60\%$ \\
            \rowcolor{grisclair} Précision rejets & Hebdomadaire & $< 95\%$ \\
            Montant erreurs FP & Hebdomadaire & $> X$ MAD \\
            \rowcolor{grisclair} Drift des features & Mensuelle & Alerte automatique \\
            \bottomrule
        \end{tabular}
    \end{center}
    
    \vspace{0.5cm}
    
    \begin{itemize}
        \item \faSync\ Réentraînement trimestriel recommandé
        \item \faChartLine\ Dashboard de monitoring en temps réel
        \item \faUsers\ Échantillonnage aléatoire pour audit qualité
    \end{itemize}
\end{frame}

% =============================================================================
% CONCLUSION
% =============================================================================
\section{Conclusion}

\begin{frame}{Synthèse}
    \begin{columns}[T]
        \begin{column}{0.5\textwidth}
            \textbf{\faCheckCircle\ Points forts}
            \begin{itemize}
                \item Accuracy élevée ($\sim$94\%)
                \item Système de confiance à 3 zones
                \item Gains financiers significatifs
                \item Traçabilité des décisions
                \item Modèle explicable
            \end{itemize}
        \end{column}
        
        \begin{column}{0.5\textwidth}
            \textbf{\faExclamationTriangle\ Points d'attention}
            \begin{itemize}
                \item Certaines familles à risque
                \item Monitoring nécessaire
                \item Formation des équipes
                \item Gestion du changement
            \end{itemize}
        \end{column}
    \end{columns}
    
    \vspace{0.5cm}
    
    \begin{successbox}
        \centering
        \Large\textbf{Le modèle est prêt pour un déploiement en pilote}
    \end{successbox}
\end{frame}

\begin{frame}{}
    \begin{center}
        \vspace{2cm}
        {\Huge\textbf{Merci de votre attention}}
        
        \vspace{1cm}
        
        {\Large Questions ?}
        
        \vspace{2cm}
        
        \textcolor{gris}{\faEnvelope\ contact@banque.ma}
    \end{center}
\end{frame}

% =============================================================================
% ANNEXES
% =============================================================================
\appendix

\begin{frame}{Annexe: Glossaire}
    \small
    \begin{tabular}{ll}
        \toprule
        \textbf{Terme} & \textbf{Définition} \\
        \midrule
        Accuracy & \% de prédictions correctes \\
        Précision & \% de positifs prédits qui sont vrais \\
        Recall & \% de vrais positifs correctement identifiés \\
        F1-Score & Moyenne harmonique précision/recall \\
        Faux Positif (FP) & Prédit fondée, réellement non fondée \\
        Faux Négatif (FN) & Prédit non fondée, réellement fondée \\
        XGBoost & Algorithme de gradient boosting \\
        Optuna & Framework d'optimisation hyperparamètres \\
        \bottomrule
    \end{tabular}
\end{frame}

\end{document}
